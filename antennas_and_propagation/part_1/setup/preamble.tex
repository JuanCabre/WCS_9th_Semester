%%%%%    PREAMBLE     %%%%%
% Basics
\documentclass[11pt,twoside]{report}
\usepackage{type1cm}
\usepackage[utf8]{inputenc}
\usepackage[english]{babel}
\usepackage{a4}
\usepackage{float}
\usepackage{lastpage}
\usepackage{amsmath} 
\usepackage[section]{placeins} % Package til \FloatBarrier så man kan styre floats ([section] inkludere automatisk \FloatBarrier i alle sections)
\usepackage[table]{xcolor}
\usepackage{umoline}		   % Package til \Overline{text} .. husk case sensitive !
\usepackage[small]{caption}
%\usepackage{microtype}

%farve til tabler
\usepackage[table]{xcolor}
\usepackage{colortbl}
%\usepackage{kpfonts}

% Indhold -> Indholdsfortegnelse, Bilag -> Appendiks
\addto\captionsdanish{
	\renewcommand\appendixname{Appendix}
	\renewcommand\contentsname{Table of contents}
}
%\usepackage{alnumsec} % Giver mulighed for at bruge \alnumsecstyle{L}, se http://ctan.mackichan.com/macros/latex/contrib/alnumsec/alnumsec.pdf
%\surroundLetter{}{}
\usepackage{appendix}

% Fjerner mellemrum efter , i equations
\usepackage{icomma}

% Giver mulighed for at inkludere pdf'er
%\usepackage{pdfpages}

% Overhold standarder for SI-enheder
\usepackage[%
    %decimalsymbol=comma,                   % Komma som tusindtalsseperator (not anymore!)
    per-mode=fraction,                     % Brug fraction ved fx \meter\per\second (ellers bruger den ms^{-1})
    exponent-product=\cdot,                % Brug \cdot ved videnskabelig notation, fx 21\cdot 10^{3} (ellers bruger den \times)
    complex-root-position = before-number, % Sæt kompleks-del foran tallet
    output-complex-root = \text{j},        % Brug j for kompleks notation
    %output-exponent-marker = \text{e}     % Brug 21e3 i stedet for 21\cdot 10^3
    group-digits = true,                    % Lille mellemrum som tusindtalsseperator
    binary-units = true
    ]{siunitx}
% ex:
%  $\SI{10}{\kilo\ohm}$ svarer til $10\:\text{k}\Omega$
%  Lille si er kun til enheder, fx: \si{\kilo\ohm}
\newcommand{\SIf}[2]{ % \SI med lille fraction (god til in-line stuff)
    %\SI[fraction-function=\slfrac]{#1}{#2}
    \SI[per-mode=symbol]{#1}{#2}
}

% Giver mulighed for at anvende Decibel i SI funktionen
\DeclareSIUnit{\decibel}{dB}
\DeclareSIUnit{\decibelm}{dBm}
\DeclareSIUnit{\decibeli}{dBi}

% Giver adgang til \singlespacing, \doublespacing, og \onehalfspacing
\usepackage{setspace}
% Halvanden linjeafstand (se også software.tex)
\onehalfspacing

% Fjern indents!
%\setlength{\parindent}{0in}

% Afstand mellem caption og teksten nedenunder
%\setlength{\belowcaptionskip}{10pt}
%\setlength{\textfloatsep}{5pt plus 1.0pt minus 2.0pt}
%\setlength{\floatsep}{5pt plus 1.0pt minus 2.0pt}
%\setlength{\intextsep}{5pt plus 1.0pt minus 2.0pt}
% Standard in articles:
%\textfloatsep: 20.0pt plus 2.0pt minus 4.0pt;
%\floatsep: 12.0pt plus 2.0pt minus 2.0pt;
%\intextsep: 12.0pt plus 2.0pt minus 2.0pt.
%\setlength{\abovecaptionskip}{-2ex}
%\setlength{\belowcaptionskip}{-2ex}

% Margins
\usepackage{vmargin}
%\setmargrb{2cm}{1cm}{3cm}{2cm} % Ligesom nedenstående, bare til PS
%\setmargrb{2cm}{1cm}{3cm}{2cm} % Ligesom udskrift, bare med plads til warnings i margin
\setmargrb{2.5cm}{3cm}{2.5cm}{2cm} % Optimal til udskrift! {Venstre/Ryg}{Top}{Højre/Ud}{Bund}

% Billeder
\usepackage{graphicx}
%\usepackage{epstopdf}
\graphicspath{{figures/}}
\usepackage[usenames,dvipsnames]{pstricks}
\usepackage{epsfig}
\usepackage{pst-grad} % For gradients
\usepackage{pst-plot} % For axes
\usepackage{pst-circ} % For circuits
\usepackage{pst-func} % For functions
\psset{gridcolor=lightgray,subgridcolor=white}


% References laves til links
\usepackage{hyperref}
\hypersetup{ 				% Borders omkring links fjernes
	pdfborder = {0 0 0}
}

% Pænere headers og footers
\usepackage{fancyhdr}
\setlength{\headheight}{14pt}
\fancyhf{} % clear header and footer and let me choos myself
% NOTE: http://timmurphy.org/2010/08/07/headers-and-footers-in-latex-using-fancyhdr/
\chead{\leftmark}
\cfoot{\thepage}
% Pænere chapter headings
%\usepackage[Conny]{fncychap}
\usepackage[Lenny]{fncychap}

% Redefiner "Lenny" så der ikke kommer så meget topmargin v. chapter headings
% Originale styles: ftp://cam.ctan.org/tex-archive/macros/latex/contrib/fncychap/fncychap.sty
\makeatletter
\ChNameVar{\fontsize{14}{16}\usefont{OT1}{phv}{m}{n}\selectfont}
  \ChNumVar{\fontsize{60}{62}\usefont{OT1}{ptm}{m}{n}\selectfont}
  \ChTitleVar{\Huge\bfseries\rm}
  \ChRuleWidth{1pt}
  \renewcommand{\DOCH}{%
  	\vspace*{-3cm}                           %flytter kapitel navn op
    \settowidth{\px}{\CNV\FmN{\@chapapp}}
    \addtolength{\px}{2pt}
    \settoheight{\py}{\CNV\FmN{\@chapapp}}
    \addtolength{\py}{1pt}

    \settowidth{\mylen}{\CNV\FmN{\@chapapp}\space\CNoV\thechapter}
    \addtolength{\mylen}{1pt}
    \settowidth{\pxx}{\CNoV\thechapter}
    \addtolength{\pxx}{-1pt}

    \settoheight{\pyy}{\CNoV\thechapter}
    \addtolength{\pyy}{-2pt}
    \setlength{\myhi}{\pyy}
    \addtolength{\myhi}{-1\py}
    \par
    \parbox[b]{\textwidth}{%
    \rule[\py]{\RW}{\myhi}%
    \hskip -\RW%
    \rule[\pyy]{\px}{\RW}%
    \hskip -\px%
    \raggedright%
    \CNV\FmN{\@chapapp}\space\CNoV\thechapter%
    \hskip1pt%
    \mghrulefill{\RW}%
    \rule{\RW}{\pyy}\par\nobreak%
    \vskip -\baselineskip%
    \vskip -\pyy%
    \hskip \mylen%
    \mghrulefill{\RW}\par\nobreak%
    \vskip \pyy}%
    \vskip 20\p@}
 

  \renewcommand{\DOTI}[1]{                   % Chapters m. chapter headings
    \raggedright
    \CTV\FmTi{#1}\par\nobreak
    \vskip 20\p@} %\vskip 40\p@}             % 40->20=mindre afstand mellem chaptername og tekst

  \renewcommand{\DOTIS}[1]{                  % Chapter uden chapter headings
  	\vspace*{-3cm}                           % flytter kapitel navn op
    \raggedright
    \CTV\FmTi{#1}\par\nobreak
    \vskip -20\p@} %\vskip 40\p@}            % 40->-20=mindre afstand mellem chaptername og tekst 
\makeatother

% Nemmere multirow-cells i tabeller
\usepackage{multirow}

% Nemmere multi kolonner
\usepackage{multicol}

% Matematik
\usepackage{amsmath,amsfonts,amssymb}
\newcommand{\slfrac}[2]{\left.#1\middle/#2\right.} % frac med slash
\usepackage{xfrac}

% Placer flere figurer ved siden af hinanden
\usepackage[center]{subfigure}

% Farver på tabeller
\usepackage{colortbl}
\usepackage{array}
\definecolor{lightgray}{RGB}{220,220,220}
\definecolor{lightergray}{RGB}{250,250,250}
\definecolor{jgray}{RGB}{130,130,130}
\definecolor{darkgray}{RGB}{40,40,40}
\definecolor{darkred}{RGB}{136,0,21}
\definecolor{darkblue}{RGB}{0,4,183}
\definecolor{darkerred}{RGB}{110,0,20}
\definecolor{darkgreen}{RGB}{0,160,0}
\definecolor{darkgreen2}{RGB}{16,100,36}
\definecolor{pink}{rgb}{1.,0.75,0.8}
\definecolor{lightblue}{RGB}{153,217,234}
\definecolor{lightyellow}{RGB}{255,255,128}
\definecolor{lightred}{RGB}{255,128,128}
\definecolor{lightpurple}{RGB}{204,0,204}
\definecolor{purple}{RGB}{128,0,128}
\definecolor{sand}{RGB}{245,240,220}
\definecolor{coolblue}{RGB}{0,128,255}


\renewcommand\arraystretch{1.2} % Sæt row height i tabeller

% Giver adgang til \begin{verbatimtab}[8] som viser indents i verbatim environments
\usepackage{moreverb}
\usepackage{fancyvrb}
\usepackage{relsize} % font size som \relsize{2}, relativt til alm fontsize

% Custom commands defineres
% Nemmere figurer, syntaks:
% \fig[keepaspectratio=true,height=40mm]{image}{Teksten til billledet}{billedelabel}
\newcommand{\fig}[4][width=40mm]{
	\begin{figure}[h!]
		\centering
	    \includegraphics[#1]{#2}
	    \caption{#3}
	    \label{#4}	
		%\end{centering}
	\end{figure}
}

\newcommand{\figuc}[4][width=40mm]{
	\begin{figure}[h!]
	    \includegraphics[#1]{#2}
	    \caption{#3}
	    \label{#4}	
	\end{figure}
}

\newcommand{\pfig}[3]{
	\begin{figure}[h!]
		\centering
	    #1
	    \caption{#2}
	    \label{#3}	

	\end{figure}
}

\newcommand{\pcapfig}[4]{
	\begin{figure}[h!]
		\centering
	    #1
	    \caption[#2]{#3}
	    \label{#4}	
	\end{figure}
}

% Nemmere referencer, syntaks:
% \figref{labelpåfigur}         --> figur 3.4 (s. 23)
% \secref{labelpåsektion}       --> sektion 2.7 (s. 7)
% \chref{labelpåchapter}        --> kapitel 2 (s. 4)
% \tref{labelpåtabel}           --> tabel 7.4 (s. 45)
% \cref{labelpåkode}            --> kodeudsnit 8.3 (s. 76)
% \bilref{labelpåbilag}         --> bilag 2.3 (s. 25)
\usepackage{nameref} % Giver adgang til \nameref

%\newcommand{\figref}[1]{figure \ref{#1} (p. \pageref{#1})}
%\newcommand{\Figref}[1]{Figure \ref{#1} (p. \pageref{#1})}
%\newcommand{\secref}[1]{section \ref{#1} (p. \pageref{#1})}
%\newcommand{\Secref}[1]{Section \ref{#1} (p. \pageref{#1})}
%\newcommand{\chref}[1]{chapter \ref{#1} (p. \pageref{#1})}
%\newcommand{\tref}[1]{table \ref{#1} (p. \pageref{#1})}
%\newcommand{\cref}[1]{code snippet \ref{#1} (p. \pageref{#1})}
%\newcommand{\bilref}[1]{annex (Se oversigt, chap. \ref{#1} (p. \pageref{#1}))}
%\newcommand{\equref}[1]{equation \ref{#1} (p. \pageref{#1})}
%\newcommand{\apref}[1]{appendix \ref{#1} (p. \pageref{#1})}
\newcommand{\figref}[1]{figure \vref{#1}}
\newcommand{\Figref}[1]{Figure \Vref{#1}}
\newcommand{\secref}[1]{section \vref{#1}}
\newcommand{\Secref}[1]{Section \Vref{#1}}
\newcommand{\chref}[1]{chapter \ref{#1}}
\newcommand{\tref}[1]{table \vref{#1}}
\newcommand{\cref}[1]{code snippet \vref{#1}}
\newcommand{\anref}[1]{annex (See list, chap. \ref{#1} (p. \pageref{#1}))}
\newcommand{\equref}[1]{equation \vref{#1}}
\newcommand{\apref}[1]{appendix \vref{#1}}
\newcommand{\afigref}[1]{figure \vref{#1}}
\newcommand{\aafigref}[1]{figure \ref{#1}}
\newcommand{\asecref}[1]{section \vref{#1}}
\newcommand{\atref}[1]{table \vref{#1}}
\newcommand{\acref}[1]{code snippet \vref{#1}}
\newcommand{\achref}[1]{chapter \ref{#1} (p. \pageref{#1})}
\newcommand{\aequref}[1]{\vref{#1}}


% Funktion til at få latex til kun at medtage side tal hvis referancen er på en anden side end den der refereres fra
\usepackage{varioref}
\makeatletter
\vref@addto\extrasenglish{%
  \def\reftextfaceafter{(p.~\thevpagerefnum)}
  \def\reftextfacebefore{(p.~\thevpagerefnum)}
  \def\reftextafter{(p.~\thevpagerefnum)}
  \def\reftextbefore{(p.~\thevpagerefnum)}
	\def\reftextfaraway#1{(p.~\thevpagerefnum)}
  \def\reftextcurrent{}
}
\makeatother

% Itemize uden mellemrum mellem linjer
\newenvironment{pitemize}{
\begin{itemize}
  \setlength{\itemsep}{1pt}
  \setlength{\parskip}{0pt}
  \setlength{\parsep}{0pt}
}{\end{itemize}}

% Enumrate uden mellemrum mellem linjer
\newenvironment{penumrate}{
\begin{enumerate}
  \setlength{\itemsep}{1pt}
  \setlength{\parskip}{0pt}
  \setlength{\parsep}{0pt}
}{\end{enumerate}}

% Kommentarer i margen (noter til forfattere)
\usepackage{xkvltxp}
\usepackage[draft]{fixme}
%\usepackage{fixme}       % Skjul margin kommentarer (til udskrift)
% Nemmere kommentarer i margin, syntaks:
%  \note{DitNavn}{Din note}{teksten, noten skal hæftes ved}
% Eks.:
%  Dette er en \note{Tausen}{Hov hov! Du har vist byttet om på kage og gulerødder}{Jeg kan godt lide kage - men gulerødder er nu bedre.} haha - hilsen jesper :D
\newcommand{\note}[3]{
	\fxnote*[author=#1,footnote,nomargin]{#2}{#3}
}

% Til kode-eksempler
\usepackage{color}                        % Package til \color-kommandoen
\usepackage{listings}                     % Package til kodeeksempler
% Caption customization:
%\usepackage{xcolor}
%\usepackage{courier}
%\usepackage{caption}
%\DeclareCaptionFont{white}{\color{white}}
%\DeclareCaptionFormat{listing}{\colorbox{gray}{\parbox{\textwidth}{#1#2#3}}}
%\captionsetup[lstlisting]{format=listing,labelfont=white,textfont=white}
\newcommand{\code}[2]{
  \FloatBarrier
  \lstinputlisting[#1]{#2}
  \FloatBarrier
}

%Ny command til tabler:
%\ptable[scalebox værdi (hvis den bare skal være default skal man ikke have de firkantede parenteser med)]{søjlernes formatering}{
%  tablen uden øverste \hline
%}{caption tekst}{label tekst}
\newcommand{\ptable}[5][0.75]{
  \begin{table}[h!]
    \centering
    %  \rowcolors{2}{lightergray}{}
      \scalebox{#1}{
        \begin{tabular}{#2}
          \hline
          \rowcolor[gray]{0.8}#3
        \end{tabular}
      }
      \caption{#4}
      \label{#5}
    %\end{center}
  \end{table}
}
% Tabeller med notes under, alt andet ligesom ovenstående
% Noterne specificeres som en itemize som sidste parameter:
% \ntable[scalebox]{column format}{table uden første \hline}{caption}{label}{notes}
% ex:
% \ntable{|m{5cm}|}{
%   \textbf{Example table with notes}
%   \hline
%   \hline
%   Row 1\tnote{1} \\
%   \hline
%   Row 2\tnote{2} \\
%   \hline
%   Row 3 \\
%   \hline
% }{
%   \item[1] This is the first row
%   \item[2] And this is the second
% }
\usepackage{threeparttable}
\newcommand{\ntable}[6][0.65]{
  \begin{table}[h!]
    \centering
      \scalebox{#1}{
        \begin{threeparttable}
          \rowcolors{2}{lightergray}{}
          \begin{tabular}{#2}
          #3
          \end{tabular}
          \begin{tablenotes}
            #6
          \end{tablenotes}
        \end{threeparttable}
      }
      \caption{#4}
      \label{#5}
 %   \end{center}
  \end{table}
}

%Ny command til tabler med multirow:
%\ptablemr[scalebox værdi (hvis den bare skal være default skal man ikke have de firkantede parenteser med)]{søjlernes formatering}{
%  tablen uden øverste \hline
%}{caption tekst}{label tekst}
%Derefter skal man selv ind og sætte farve i rækkerne
\newcommand{\ptablemr}[5][0.8]{
  \begin{table}[h!]
    \centering
      \scalebox{#1}{
        \begin{tabular}{#2}
          \hline
          \rowcolor[gray]{0.8}#3
        \end{tabular}
      }
      \caption{#4}
      \label{#5}
  %  \end{center}
  \end{table}
}

%Ny command til tabler til mr. t:
%\ptable[scalebox værdi (hvis den bare skal være default skal man ikke have de firkantede parenteser med)]{søjlernes formatering}{
%  tablen 
%}{caption tekst}{label tekst}
\newcommand{\ptablemrt}[5][0.65]{
  \begin{table}[h!]
    \centering
      \rowcolors{2}{lightergray}{}
      \scalebox{#1}{
        \begin{tabular}{#2}
          #3
        \end{tabular}
      }
      \caption{#4}
      \label{#5}
 %   \end{center}
  \end{table}
}
    
\def\lstlistingname{Code snippet}           % Definerer hvad der står foran et stykke kodes caption
\lstset{
	basicstyle=\footnotesize\ttfamily,    % Lille skrifttype
	keywordstyle=\color{blue}\bfseries,   % Keywords blå og bold
	commentstyle=\color[RGB]{34,139,34},  % Default comments mørkegrøn
	showstringspaces=false,               % Ingen symbol for mellemrum i strings
	numbers=left,                         % Linjenumre til venstre
	numberstyle=\tiny\color{darkgray},    % Små tal på linjenumre med farve skrift
	numbersep=5pt,                        % Afstand fra linjenummer og ind til kode
	backgroundcolor=\color{lightergray},  % Bg farve
	tabsize=2,                            % Indenteringer = 4 spaces
	columns=fixed,                   	  % Kan give problemer med bredde på bogstaver men skulle ikke da vi bruger ttfamily
	breaklines=true,                      % Deler en for lang linje over to linjer
	frame=tb,                		      % Styrer hvor streger skal placeres
	captionpos=t,                         % Caption til kode under og over kodeeksemplet
	rulecolor=\color{black},			  % Farven på frame
	escapeinside={(*@}{@*)},               % Giver mulighed for at lave en (*@\label{label}@*), på en kodelinje,
%										    så man kan referere til linjen
    literate={~}{$\sim$}1 {^}{$\wedge$}1,
}
\lstdefinelanguage{bascom} {              % Definition af BASCOM-language
	classoffset=4,	
	morekeywords={\$regfile,\$crystal,Config, Output, Input, Timer1, Timer0, Do, Loop, If, Then, End, Sub, Wait, Waitms, Declare, As, Int, Word, Byte, Call, And, Or, Else, Until, Goto, Alias, Dim},
		keywordstyle=\color[RGB]{0,0,128}\bfseries,classoffset=3,
	morekeywords={PORTA, PORTB, TCCR1A, TCCR1B, TCCR1C, TCCR1D, TCNT1, OCR1A, OCR1B, OCR1C, OCR1D, PINA, PINB, PINC},
		keywordstyle=\color[RGB]{128,0,0},classoffset=2,
	morekeywords={=, \&, <, >, +, -, *, /, .},
		keywordstyle=\color[RGB]{255,0,0},classoffset=1,
	sensitive=false,
	morecomment=[l]{'},
	commentstyle=\color[RGB]{34,139,34}
}
\lstdefinelanguage{arduino} {              % Definition af ARDUINO-language
	classoffset=4,
	morekeywords={char,int,void,long,
			pinMode,random,available,read,print,millis,digitalWrite,digitalRead,analogWrite,analogRead,delay,
			for,while,switch,break,if,else,bitRead,print,println,begin,return,float,cos,sin,pow,sqrt,bitClear,bitSet,
			true,false},
		keywordstyle=\color[RGB]{204,102,0},classoffset=3,
	morekeywords={setup,Serial,Serial1,Serial2,loop},
		keywordstyle=\color[RGB]{204,102,0}\bfseries,classoffset=2,
	morekeywords={HIGH,LOW,OUTPUT,INPUT},
		keywordstyle=\color[RGB]{0,102,153},classoffset=1,
	sensitive=false,
	morecomment=[l]{//},
	stringstyle=\color[RGB]{0,0,255},
	morestring=[b]",
	morestring=[b]',
	commentstyle=\color[RGB]{50,50,50}
}
\lstdefinelanguage{csharp} {              % Definition af C#-language
	classoffset=3,
	% Variabletypes, keywords
	morekeywords={char,int,void,long,object,string,true,char,false,using,class,
			this,delegate,partial,namespace,
			for,while,switch,break,if,else,new,try,catch,private,public,\#region,\#endregion},
		keywordstyle=\color[RGB]{0,0,255},classoffset=2,
	% Classes
	morekeywords={EventArgs,SerialPortController,StringBuilder,SerialDataReceivedEventHandler,MySqlConnection,
			Exception,MySqlCommand,MySqlDataReader,ConnectionState,PointPairList,ZedGraphControl,Color,
			SymbolType,GraphPane,LineItem,Fill,Size,Point,MethodInvoker,Convert,FormClosedEventArgs,EventArgs,
			SerialPort,StopBits,Parity,Form,Form1},
		keywordstyle=\color[RGB]{43,145,175},classoffset=1,
	sensitive=true,
	morecomment=[l]{//},
	stringstyle=\color[RGB]{163,21,21},
	morestring=[b]",
	morestring=[b]',
	commentstyle=\color[RGB]{0,128,0}
}
\lstdefinelanguage{vhdl} {              % Definition af VHDL-language
	classoffset=4,
	morekeywords={library,use,all,entity,generic,map,architecture,of,is,downto,others,then,if,port,signal,elsif,else,when,for,while,
	              end,process,begin,or,and,not,xor,\&,constant,wait,for,nor,nand,in,out,type,array,with,select,case},
		keywordstyle=\color[RGB]{255,0,0},classoffset=3,
	morekeywords={std_logic,std_logic_vector,to_integer,unsigned,integer,time,natural,1,2,3,4,5,6,7,8,9,0},
		keywordstyle=\color[RGB]{0,145,185},classoffset=2,
    morekeywords={=, <, >, +, -, *, /, .},
		keywordstyle=\color[RGB]{0,0,255},classoffset=1,
	sensitive=true,
	morecomment=[l]{--},
	stringstyle=\color[RGB]{0,145,185},
	morestring=[b]",
	commentstyle=\color[RGB]{0,128,0}
}
\lstdefinelanguage{psm} {              % Definition af PSM(ASM)-language
	classoffset=4,
	morekeywords={and,or,xor,out,in,load,sub,add,comp,call,jump,ret,store,fetch,test,sr0,sr1,sl0,sl1},
		keywordstyle=\color[RGB]{0,0,255},classoffset=3,
	morekeywords={equ,dsout,dsin},
		keywordstyle=\color[RGB]{108,0,117},classoffset=2,
    morekeywords={\$,0,1,2,3,4,5,6,7,8,9},
		keywordstyle=\color[RGB]{0,232,116},classoffset=1,
	sensitive=true,
	morecomment=[l]{;},
	stringstyle=\color[RGB]{0,145,185}
%	morecomment=[l]{\$},
%    commentstyle=\color[RGB]{0,232,116}
}
% TOC settings
\setcounter{tocdepth}{1} % Begræns table of contents til kun 2 niveauer
%\setcounter{tocdepth}{10}
\addtocontents{toc}{\protect\thispagestyle{empty}} % Fjern sidetal fra TOC

% Flowcharts
\usepackage{tikz} 
\newcommand*\circled[1]{\tikz[baseline=(char.base)]{ % circle om tekst
            \node[shape=circle,draw,inner sep=2pt] (char) {#1};}}
\usetikzlibrary{shapes,arrows}
% Define block styles
\tikzstyle{decision} = [diamond, draw=gray, aspect=2, fill=lightgray, text width=2cm, text badly centered, node distance=3cm, inner sep=0pt,minimum height=1cm]
%\tikzstyle{decision} = [regular polygon, regular polygon sides=4, shape border rotate=45, draw, fill=lightgray, text width=1.5cm, text badly centered, node distance=3cm, inner sep=0pt,minimum height=1cm]
\tikzstyle{block} = [rectangle, draw=gray, fill=lightgray, text width=3cm, text centered, minimum height=1cm]
\tikzstyle{endpoint} = [ellipse, draw=gray, fill=lightgray, text width=2cm, text centered, minimum height=1cm]
\tikzstyle{line} = [draw, thick, -latex]

\newcommand{\flowcirc}[1]{
  \psovalbox[fillstyle=solid,fillcolor=lightgray,linecolor=gray]{#1}
}
\newcommand{\flowbox}[1]{
  \psframebox[fillstyle=solid,fillcolor=lightgray,linecolor=gray]{#1}
}
\newcommand{\flowdiamond}[1]{
  \psdiabox[fillstyle=solid,fillcolor=lightgray,linecolor=gray]{#1}
}
\usepackage[square]{natbib}

%%% Stops figures and stuff to float to next section %%%
\usepackage[section]{placeins}
\let\OLDsubsection=\subsection
\renewcommand{\subsection}[1]{\FloatBarrier\OLDsubsection{#1}}
\let\OLDsubsubsection=\subsubsection
\renewcommand{\subsubsection}[1]{\FloatBarrier\OLDsubsubsection{#1}}

% Glossaries
\usepackage[acronym,xindy]{glossaries}
\makeglossaries
\usepackage[xindy]{imakeidx}
\makeindex

\newcommand{\eqdef}{\overset{\mathrm{def}}{=\joinrel=}}