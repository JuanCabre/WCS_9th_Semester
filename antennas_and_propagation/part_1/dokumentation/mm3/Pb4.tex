\section{Problem 4}
\textit{Alfa equal 55 degrees is claimed to be a good references – why? B) and why then measure in both 0, 90, 45 and -45 degrees and not 55 degrees?}

To make an evaluation of the environment it is needed to minimize the effect produced by the different polarizations in the received waves in such area. To reduce this effect it can be proved that for a half-length dipole, the mean effective gain (MEG) is the same independently of the cross polarization ratio $XPR$, when said dipole is inclined 55 degrees in its elevation angle.

When said dipole is inclined, it defines a plane (called antenna inclination plane in the bibliography). Subsequently, to evaluate the environment in which the antenna is located, the dipole is rotated in its inclination plane and the power measurements are performed.