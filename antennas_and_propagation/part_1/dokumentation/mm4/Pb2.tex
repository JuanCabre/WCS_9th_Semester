\section{Problem 2}
\textit{Figure 2.59 shows a polarization diversity configuration having radiation patterns as shown in figure 2.60 as a function of alpha. For alpha equal 0, pure polarization diversity is not obtained due to the problem on the Z-axis but is that the only problem?} 

To consider pure diversity we need not just that the polarization is appropriate and we receive in each dipole the waves of horizontal and vertical polarization exclusively but to also receive them with approximately the same power.

In the problem considering the cross-dipole antennas represented in Figure 2.59, for $\alpha=0$ the dipole number 2, besides the desirable HP is gathering a small part of the VP what would be the first impediment to achieve pure diversity. But at the same time, as it is observed from the superposition of the radiation patterns for both dipoles in figure 2.60, the gain of both dipoles is similar just in a particular direction (Y axis). 

\textit{b) if you are asked to make a polarization diversity antenna for a base station (e.g. using array techniques) how should the radiation pattern look like?}

In order to obtain the higher polarization diversity we would design an antenna with high directivity towards the Y axis, what can be accomplished with an array of cross-dipole antennas.