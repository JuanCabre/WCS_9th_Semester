\section{Exercise 1-2} \label{sec:mm7_Ex2}
\textit{Consider $K$ CDMA users that are transmitting to a base station (BS). The received SINR for k-th user is given by the following expression:}
\begin{flalign}
&& SINR_{k} =& \frac{G\,Q_k}{\sum\limits_{n\neq k}Q_n+N_0 W}  & \label{eq:SINRmm7EX2}
\end{flalign} 

\noindent\textit{where Qn is the received power of the n-th user at the BS, $N_0$ is the noise spectral density, $W$ is the total bandwidth used in the system and $G$ is the processing gain.}

\subsection{a)}
\textit{Assume that there are only $K=2$ users in the system. The received powers of the users are such that}
\begin{flalign}
&& \frac{Q_1}{N_0 W} =& 20 & [dB]  & \label{eq:Q1} \\
&& \frac{Q_2}{N_0 W} =& 15 & [dB]  & \label{eq:Q2}
\end{flalign} 
\noindent\textit{The rate at which the user k can communicate is selected to be}
\begin{flalign}
&& r_{k} =& W \log_2(1+SINR_k) & [bps]  & 
\label{eq:rate}
\end{flalign} 
\noindent\textit{If $W=10$ MHz, find the rates at which each user can communicate with the BS. The processing gain is $G=10$.} \\

From equations \eqref{eq:Q1} and \eqref{eq:Q2} we get the received power for each user at the base station: $Q_{1}= 20[dB]N_{0}W$, which in linear would give $Q_{1}=100N_{0}W$. Same for the second user's received power: $Q_{2}=31.6N_{0}W$. \\

By using \equref{eq:rate}, we can compute the rate at which each user can communicate. Therefore, the rates for the two users have been computed as in \equref{}.
\begin{flalign}
&& r_{1} =& 10 [MHz]log_{2}1+\left( \frac{GQ_{1}}{Q_{2}+N_{0}W}\right)  & [dB]  & \label{eq:Q1} \\
&& \frac{Q_2}{N_0 W} =& 15 & [dB]  & \label{eq:Q2}
\end{flalign} 


\subsection{b)}
\textit{Assuming that all parameters except the processing $G$ are fixed, arbitrary value of $K$, and the minimal required SINR by each user is equal to $\beta$, find the minimal $G$ that will enable each user to achieve the minimal required SINR.}

\textit{Assuming that all parameters except the processing $G$ are fixed, arbitrary value of $K$, and the minimal required SINR by each user is equal to $\beta$, find the minimal $G$ that will enable each user to achieve the minimal required SINR.}\\

In this exercise the aim is to minimize G. In order to do this we start with the equation for the SINR, also shown in \equref{eq:SINRmm7EX2}, and equivalent that to $\beta$:
\begin{flalign}
&& \frac{G\,Q_k}{\sum\limits_{n\neq k}Q_n+N_0 W} &= \beta & \label{eq:SINRmm7EX2_b}
\end{flalign} 

For a given value of $Q$ here denoted $Q_w$ the resulting $SINR_W$ is equivalent to $\beta$. Using this we can rewrite \equref{eq:SINRmm7EX2_b} to an expression for G. This will result in the following:
\begin{flalign}
&& G &= \frac{\beta\left(\sum\limits_{n\neq k}Q_n+N_0 W \right)}{Q_W} & 
\end{flalign} 