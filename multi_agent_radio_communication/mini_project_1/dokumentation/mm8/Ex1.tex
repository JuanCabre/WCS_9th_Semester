\section{Exercise 1} \label{sec:mm8_Ex1}
\textit{A list of parameters of a Wireless LAN air-interface (IEEE 802.11a) using OFDM is given below:
\begin{itemize}
 \item 64 subcarriers, 48 for data, 4 for pilots and 12 null subcarriers.
 \item Symbol duration 4 $\mu$s.
 \item Cyclic prefix 0.8 $\mu$s.
 \item Modulation: BPSK, QPSK, 16-QAM, 64-QAM
 \item Convolutional coding with rate 1/2 ; 2/3 ; 3/4
 \item System bandwidth 20MHz
 \item Bit rates of 6, 12, 18, 24, 36, 48 and 54Mbps
\end{itemize}
Calculate the following parameters:}


\subsection*{1) \textit{FFT time-period}}

For OFDM the FFT is taken over the 64 subcarriers, either the time period can be calculated from equation \ref{eq:mm8_ex2_q1} or from equation \ref{eq:mm8_ex1_q1}

\begin{flalign} \label{eq:mm8_ex2_q1}
 && T_{symbol} =& T_{CP} + T_{FFT} & \\
 && T_{FFT} =& T_{symbol} - T_{CP} & \\
\end{flalign}
Where $T_{symbol}$ = 4 $\mu$s is the time duration of the symbol, and $T_{CP}$ = 0.8 $\mu$s is the time duration of the cyclic prefix.
\begin{flalign}
&& T_{FFT}=&N_{subcarriers} \cdot \frac{1}{BW_{system}} &
\end{flalign}

\begin{flalign}
 && T_{FFT} =& 3.2 [\SI{}{\micro\second}] & \label{eq:mm8_ex1_q1}
\end{flalign}

\subsection*{2) \textit{Sampling frequency}}
The sampling frequency for a baseband signal is 
\begin{flalign}
&& F_{S} =& \frac{N_{sub}}{T_{FFT}} &
\end{flalign}
$N_{sub}$ = 64 and $T_{FFT}$ from \equref{eq:mm8_ex1_q1}
\begin{flalign}
&& F_{S}=& 20  [\SIf{}{\mega\hertz}]&
\end{flalign}

\subsection*{3) \textit{Sampling duration}}

The sampling duration is equal to the time-period of the symbol $T_{symbol}$.
\begin{flalign}
&& T_{S} =& T_{symbol} &
\end{flalign}

\subsection*{4) \textit{Number of samples in the guard interval}}

With a sampling frequency of 20 [MHz] and a duration of the guard interval of $T_{CP} = 0.8\mu$s
\begin{flalign}
&& NS_{GI} =& F_{S} \cdot T_{CP} &\\
&& =& 16 & 
\end{flalign}

\subsection*{5) \textit{Subcarrier frequency spacing}}

The spacing between the subcarriers should correspond to:
\begin{flalign}
 && F_{spacing} =& \frac{1}{T_{FFT}} & \\
 && =& 312.5 [kHz] &
\end{flalign}


\subsection*{6) \textit{For different data rates, different coding scheme and different combination of modulation scheme, calculate}}


\subsubsection*{a)} 
\textit{Coded bits per subcarriers}
The coded bits,$C_{b}$, can be calculated by 
\begin{flalign}
&& C_{bsc} =& log_{2}(M)&
\end{flalign}
Where M is the order 
\\
For BPSK the coded bits per subcarrier is: 
\begin{flalign}
&& C_{bsc(BPSK)} =& 1&
\end{flalign}
For QPSK the coded bits per subcarrier is: 
\begin{flalign}
&& C_{bsc(QPSK)} =& 2&
\end{flalign}
For 16QAM the coded bits per subcarrier is: 
\begin{flalign}
&& C_{bsc(16QAM)} =& 4&
\end{flalign}
For 64QAM the coded bits per subcarrier is: 
\begin{flalign}
&& C_{bsc(64QAM)} =& 6&
\end{flalign}

\subsubsection*{b)} 
\textit{Coded bits per OFDM symbol}
\begin{flalign}
&& C_{bsy} =& N_{data carriers} \cdot log_{2}(M)&
\end{flalign}
For BPSK the coded bits per OFDM symbol is: 
\begin{flalign}
&& C_{bsy(BPSK)} \cdot N_{dc}=& 1 \cdot 48& \\
&& =& 48 &
\end{flalign}
For QPSK the coded bits per OFDM symbol is: 
\begin{flalign}
&& C_{bsy(QPSK)} \cdot N_{dc}=& 2 \cdot 48& \\
&& =& 96 &
\end{flalign}
For 16QAM the coded bits per OFDM symbol is: 
\begin{flalign}
&& C_{bsy(16QAM)}\cdot N_{dc} =& 4 \cdot 48& \\
&& =&  192& 
\end{flalign}
For 64QAM the coded bits per OFDM symbol is: 
\begin{flalign}
&& C_{bsy(64QAM)}\cdot N_{dc} =& 6 \cdot 48& \\
&& =& 288 &
\end{flalign}

\subsubsection*{c)} 
\textit{Data bits per OFDM symbol}
The data bits is calculated based on the number of coded bits per OFDM symbol and the coding rate. The coding rate can be:r = 1/2, 2/3, 3/4.
For BPSK the data bits per OFDM symbol is: 
\begin{flalign}
&& C_{bsy(BPSK)} \cdot N_{dc} \cdot r =& 1 \cdot 48 \cdot r& \\
&& (r=1/2)=& 24 & \\
&& (r=2/3)=& 32& \\
&& (r=3/4)&  36&
\end{flalign}

For QPSK the data bits per OFDM symbol is: 
\begin{flalign}
&& C_{bsy(QPSK)} \cdot N_{dc} \cdot r =& 2 \cdot 48 \cdot r& \\
&& (r=1/2)=& 48 & \\
&& (r=2/3)=& 64& \\
&& (r=3/4)&  72&
\end{flalign}

For 16QAM the data bits per OFDM symbol is: 
\begin{flalign}
&& C_{bsy(16QAM)} \cdot N_{dc} \cdot r =& 4 \cdot 48 \cdot r& \\
&& (r=1/2)=& 96 & \\
&& (r=2/3)=& 128& \\
&& (r=3/4)&  144&
\end{flalign}

For 64QAM the data bits per OFDM symbol is: 
\begin{flalign}
&& C_{bsy(64QAM)} \cdot N_{dc} \cdot r =& 4 \cdot 48 \cdot r& \\
&& (r=1/2)=& 144 & \\
&& (r=2/3)=& 192& \\
&& (r=3/4)&  216&
\end{flalign}