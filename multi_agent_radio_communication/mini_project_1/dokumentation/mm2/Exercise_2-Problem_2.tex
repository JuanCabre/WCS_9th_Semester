\section{Exercise 2 - Problem 2}
\textit{To waterfill or not?}\\

\textit{Consider a 2x2 MIMO channel with ZMCSCG elements with unit variance. Develop your code for a square MIMO system of arbitrary number of antennas (and not just 2) since you will need it later in the problem. How does channel knowledge affect the Ergodic Capacity and the 10 percent Outage Capacity at low and high SNR values.} \\

\textit{Now increase the number of antennas to 4, 6 and 8 (remember the system is still MxM). How does channel knowledge affect ergodic capacity at 10 dB SNR with an increase in the number of antennas.}\\

ZMCSCG stands for Zero-Mean Circulant Symmetric Complex Gaussian. \\

Using multiple transmit and receive antennas will help increase the capacity of the MIMO system. This can be seen from the capacity values obtained in Matlab, where the capacity formula is $C=C+log_{2}(1+SNR \lambda ^{2})$. The channel capacity is also higly related to the correlation between transmit and receive antennas. The capacity is reduced as the correlation becomes higher. .......\\

Another statistical notion of the channel capacity is the outage channel capacity. It is defined as $P_{out}(R)=Pr(C(H)<R)$. \\

The maximum outage capacity can be defined as the maximum rate that can be maintained in all channel states with some probability of outage (no data transmission).
