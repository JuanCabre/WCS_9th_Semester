\section{Problem C}
\textit{REDUCING MAXIMUM PATH LOSS. For the same given coverage area, compare the maximum path loss of a single antenna system, $PL_{max^1}$, to the maximum path loss of an M-antenna distributed system, $PL_{max^M}$, and express the reduction in the dynamic range. Define path loss at the cell boundary, $d_c$, of the single-antenna system (as in the boxed expression above). State the implications of the relation.}\\

The maximum pathloss is given by:
\begin{flalign}
&& PL_{max} \equiv & \frac{P_R}{P_{d_c}} = C\left(\frac{A_c}{\pi}\right)^{\sfrac{\gamma}{2}} & \label{eq:MaximumPathLoss}
\end{flalign}  

where $PL_{max}$ is the maximum pathloss, $P_R$ is the reference power which is equivalent to the transmitted power and $P_{d_c}$ is the minimum acceptable downlink power level at the cell boundary which is at the distance $d_c$ from the cell center (Radius of coverage area). $A_c$ is the coverage area of the cell and $\gamma$ is the pathloss exponent. C is a scaling constant. \\

From \equref{eq:MaximumPathLoss} it can be seen that there is a direct relation to the area of the cell. Therefore the relation between number of antennas ($M$) and the area ($A_c$) found in \secref{sec:mm3_PbA} and shown in \equref{eq:ExpressionForAcWithM} can be used here. 

