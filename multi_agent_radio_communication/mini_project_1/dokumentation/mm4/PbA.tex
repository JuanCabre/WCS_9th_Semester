\section{Problem A - Uni directional : invariance1} \label{sec:mm4_PbA}

\subsection{Question 1}
\textit{Consider an antenna with power pattern $a=a1+a2$ with $a1=sin(\Phi)^{30}$ from 53 to 128 deg. and $a2=0.25*cos(\Phi - \frac{\pi}{4})^{80}$ from 24 to 66 deg, zero else where (\figref{fig:antenna_power_pattern})}.
\fig[keepaspectratio=true,width=10cm]{antenna_power_pattern.eps}{Antenna Power Pattern}{fig:antenna_power_pattern}

In \figref{fig:antenna_power_pattern} the antenna power is set to 30 dB.

\subsubsection{a) what is the approx. peak to null depth compared to main and side lobe}

Compared to the main and side lobe, from \figref{fig:antenna_power_pattern} it can be noticed that the peak to null depth is around 60 degrees at 15 dB.

\subsubsection{b) Now an user Co is positioned at 90 deg with 0dB and an interfere Io at 60 deg with -6dB. What is the approx. Co/Io?}

Relating to \figref{fig:antenna_power_pattern}, the position of the user at 90 deg with 0 dB indicates that the user's transmit power is 30 dB, the same as for the interfere. Therefore, when calculating the SIR these two will cancel out.\\

By positioning the interfere at 60 deg with -6 dB will exceed in an SIR given by the peak to null depth value found in subsection a (15 dB), to which the interfere is added. The resulting SIR will be given by $15 + 6$ = $21$ dB.
